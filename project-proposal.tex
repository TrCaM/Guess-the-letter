\documentclass[12pt,english,]{article}
\usepackage{lmodern}
\usepackage{amssymb,amsmath}
\usepackage{ifxetex,ifluatex}
\usepackage{fixltx2e} % provides \textsubscript
\ifnum 0\ifxetex 1\fi\ifluatex 1\fi=0 % if pdftex
  \usepackage[T1]{fontenc}
  \usepackage[utf8]{inputenc}
\else % if luatex or xelatex
  \ifxetex
    \usepackage{mathspec}
  \else
    \usepackage{fontspec}
  \fi
  \defaultfontfeatures{Ligatures=TeX,Scale=MatchLowercase}
\fi
% use upquote if available, for straight quotes in verbatim environments
\IfFileExists{upquote.sty}{\usepackage{upquote}}{}
% use microtype if available
\IfFileExists{microtype.sty}{%
\usepackage{microtype}
\UseMicrotypeSet[protrusion]{basicmath} % disable protrusion for tt fonts
}{}
\usepackage[margin=1in]{geometry}
\usepackage{hyperref}
\hypersetup{unicode=true,
            pdfborder={0 0 0},
            breaklinks=true}
\urlstyle{same}  % don't use monospace font for urls
\ifnum 0\ifxetex 1\fi\ifluatex 1\fi=0 % if pdftex
  \usepackage[shorthands=off,main=english]{babel}
\else
  \usepackage{polyglossia}
  \setmainlanguage[]{english}
\fi
\usepackage{graphicx}
% grffile has become a legacy package: https://ctan.org/pkg/grffile
\IfFileExists{grffile.sty}{%
\usepackage{grffile}
}{}
\makeatletter
\def\maxwidth{\ifdim\Gin@nat@width>\linewidth\linewidth\else\Gin@nat@width\fi}
\def\maxheight{\ifdim\Gin@nat@height>\textheight\textheight\else\Gin@nat@height\fi}
\makeatother
% Scale images if necessary, so that they will not overflow the page
% margins by default, and it is still possible to overwrite the defaults
% using explicit options in \includegraphics[width, height, ...]{}
\setkeys{Gin}{width=\maxwidth,height=\maxheight,keepaspectratio}
\IfFileExists{parskip.sty}{%
\usepackage{parskip}
}{% else
\setlength{\parindent}{0pt}
\setlength{\parskip}{6pt plus 2pt minus 1pt}
}
\setlength{\emergencystretch}{3em}  % prevent overfull lines
\providecommand{\tightlist}{%
  \setlength{\itemsep}{0pt}\setlength{\parskip}{0pt}}
\setcounter{secnumdepth}{0}
% Redefines (sub)paragraphs to behave more like sections
\ifx\paragraph\undefined\else
\let\oldparagraph\paragraph
\renewcommand{\paragraph}[1]{\oldparagraph{#1}\mbox{}}
\fi
\ifx\subparagraph\undefined\else
\let\oldsubparagraph\subparagraph
\renewcommand{\subparagraph}[1]{\oldsubparagraph{#1}\mbox{}}
\fi

%%% Use protect on footnotes to avoid problems with footnotes in titles
\let\rmarkdownfootnote\footnote%
\def\footnote{\protect\rmarkdownfootnote}

%%% Change title format to be more compact
\usepackage{titling}

% Create subtitle command for use in maketitle
\providecommand{\subtitle}[1]{
  \posttitle{
    \begin{center}\large#1\end{center}
    }
}

\setlength{\droptitle}{-2em}

  \title{\Huge\textbf{English Hand Written Letters Classification Using Various AI Techniques}}
    \pretitle{\vspace{\droptitle}\centering\huge}
  \posttitle{\par}
  \subtitle{COMP 4106 - Project proposal}
  \author{}
    \preauthor{}\postauthor{}
      \predate{\centering\large\emph}
  \postdate{\par}
    \date{28 February 2020}

\usepackage{float}
\usepackage{amsfonts}
\usepackage{dsfont}
\let\origfigure\figure
\let\endorigfigure\endfigure
\renewenvironment{figure}[1][2] {
    \expandafter\origfigure\expandafter[H]
} {
    \endorigfigure
}

\begin{document}
\maketitle

\newgeometry{top=1in,bottom=1in,right=0.5in,left=1in}

\newpage

\hypertarget{proposal-english-hand-written-letters-classification-using-various-ai-techniques}{%
\subsubsection{Proposal: English Hand Written Letters Classification
Using Various AI
Techniques}\label{proposal-english-hand-written-letters-classification-using-various-ai-techniques}}

Name: Tri Cao -- Student number: 100971065

\hypertarget{i.-motivation}{%
\paragraph{I. Motivation}\label{i.-motivation}}

Artificial Intelligent(AI) is a very board branch of Computer Science
with many real word applications. AI itself has many subfield, and one
such interesting topic is Computer Vision. Thanks to advance AI
technology, nowadays just a little camera on hour phone can provide a
lot of information to users. In this project, I want to take the first
step into the field by trying to recognize hand written English
characters using different AI methods and compare their performances
based on accuracy and time/space complexity.

\hypertarget{ii.-project-description}{%
\paragraph{II. Project description}\label{ii.-project-description}}

\hypertarget{general}{%
\subparagraph{1. General}\label{general}}

The main purpose of the project is to develop various AI and machine
learning techniques classify English hand written letters. Specifically,
I will develop a software which allows the user to draw English
lowercase alphabetical letters, and the program will use various AI
methods to predict the letter from 26 total in the alphabet, and compare
the performances between methods.

\hypertarget{software-structure}{%
\subparagraph{2. Software structure}\label{software-structure}}

I will develop a simple web application that allows users to draw the
hand written digits on the computer screen. In order to have the proper
complexity for the duration of the project, I will only use a 28 x 28
pixels grid to allow user to draw the letters. The web client then sends
the images to a back-end engine which contains the trained letter
recognition AI models to perform predictions.

After the predictions has been computed, the result together with
performances metrics will be returned back to the web client to be
displayed to the users.

\hypertarget{building-the-ai-models}{%
\paragraph{3. Building the AI models}\label{building-the-ai-models}}

\hypertarget{training-data}{%
\subparagraph{4. Training data}\label{training-data}}

In order to perform the predictions, training data is necessary for the
task. We need to obtain a good source hand written letter images
together with their labels (1 of the 26 character) in order to perform
the classification task. The AI/machine learning model can then use
these examples to predict the given pictures by the users

For this project, I have obtain the
\href{https://www.nist.gov/itl/products-and-services/emnist-dataset}{EMNIST}
dataset which contains 145,600 images of 28 x 28 pixels of handle
written English letters, which is big enough to train my models.

\hypertarget{using-existing-library}{%
\subparagraph{5. Using existing library}\label{using-existing-library}}

All of the models (except Neural network) will be implemented from
scratch without directly use the existing opensource implementation
available on the internet. However, to assist complex mathematical
calculations as quick performance measurements, I am going to use some
existing library listed below.

\begin{itemize}
\tightlist
\item
  AI model development language: \textbf{python}
\item
  Python library used:

  \begin{itemize}
  \tightlist
  \item
    \textbf{Numpy}: Support vector/matrix calculation in high
    dimensional array
  \item
    \textbf{Pandas}: Library that helps working with big dataset
  \item
    \textbf{emnist}: Support getting and reading the EMNIST dataset
  \item
    \textbf{sklearn}: Provide various machine learning utilities. I will
    \textbf{not} use their machine learning models to perform the
    prediction
  \item
    \textbf{tensorflow}: support building neural networks
  \end{itemize}
\end{itemize}

\hypertarget{iii.-ai-techniques}{%
\paragraph{III. AI techniques}\label{iii.-ai-techniques}}

Below is the list of possible methods that I will implement and use to
build the AI classification models. The number of methods to be
implementations can be adjusted to suit the duration of the project.

\hypertarget{nearest-neighbors-search}{%
\subparagraph{1. Nearest neighbors
search}\label{nearest-neighbors-search}}

The nearest neighbor search (NNS) is an optimization problem of finding
the point that is closest to the sets of points. In the letter
classification setting, I can map each training images into a high
dimensional vector, forming a search space for predictions. For new
images given by user, the model can find the nearest neighbors and
perform the classification using the labels from the nearest neighbors

\hypertarget{logistic-regression}{%
\subparagraph{2. Logistic Regression}\label{logistic-regression}}

Logistic Regression is a probabilistic model which uses log-odds ratio
to determine the decision boundary of the label classes. It directly
models the log-odds with a linear function, thus we can maximize the
likelihood to obtain a linear decision boundary, using gradient descent
method.

\hypertarget{linear-vector-machine-svm}{%
\subparagraph{3. Linear Vector Machine
(SVM)}\label{linear-vector-machine-svm}}

Another method that we used was Linear Support vector machine (SVM).
Linear SVM is another common classifier whose idea is to find a linear
decision boundary that represents the largest separation between to
classes, thus it increases the overall generalization. Formally, A
support-vector machine constructs a hyperplane or set of hyperplanes in
a high or infinite-dimensional space, in which a good hyperplane has the
largest distance to the nearest training-data point of any class.

\hypertarget{neural-network-with-tensorflow}{%
\subparagraph{4. Neural network (with
Tensorflow)}\label{neural-network-with-tensorflow}}

Neural network is well-known in solving these classification problem.
However, due to their complexity, I will use the tensorflow to build the
network. The purpose of using the library is to compare which model
performs the best.


\end{document}
